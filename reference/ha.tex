\documentclass[11pt,a4paper]{article}
\usepackage[margin=1in]{geometry} % full-width
    \topskip        =   20pt
    \parskip        =   10pt
    \parindent      =   0 pt
    \baselineskip   =   15pt
\usepackage{amssymb, amsfonts, amsmath}
\usepackage{bm}
\usepackage{amsmath}
\usepackage{booktabs}  % neatly formatting lines
\usepackage{threeparttable}
\usepackage{graphicx}
\usepackage{caption}
\usepackage{mathtools}
\usepackage{subfig}
\usepackage[shortlabels]{enumitem}

%Russian-specific packages
%--------------------------------------
\usepackage[T2A]{fontenc}
\usepackage[utf8]{inputenc}
\usepackage[russian]{babel}

\newcommand{\bo}[1]{\mathbf{#1}}
\newcommand{\R}{\mathbb{R}}
\newcommand{\E}{\mathbb{E}}
\newcommand{\Y}{X_i'\beta + u_i}
\newcommand{\img}[3]{
    \begin{figure*}[!hbtp]
        \centering
        \caption{#3}
        \includegraphics[scale=#2]{#1}
    \end{figure*}
}
\newcommand{\imgs}[4]{
    \begin{figure}[!hbtp]
        \centering
        \begin{minipage}{.5\textwidth}
            \centering
            \includegraphics[width=.9\linewidth]{#1}
            \captionof{figure}{#3}
        \end{minipage}
        \begin{minipage}{.49\textwidth}
            \centering
            \includegraphics[width=.9\linewidth]{#2}
            \captionof{figure}{#4}
        \end{minipage}
    \end{figure}
}

\begin{document}
\section*{Мотивация}
Хотим решать задачу по поиску оптимальных весов портфеля:
\begin{gather*}
    \frac{1}{2}\omega V \omega' \to \max \\
    \text{s.t}~~ \omega'a = 1
\end{gather*}
Выпуклая оптимизация (см. приложение) дает следующее решение:
\[
    \hat\omega = \frac{V^{-1}a}{a'V^{-1}a}
\]
Как можно видеть из формулы, необходимо оборачивать ковариационную
матрицу доходностей бумаг, из которых мы собираем портфель.
Сама по себе ковариационная матрица случайна, поскольку
мы получаем её оценки на основе данных, используя ММП.
Эмпирика показывает, что оценки весов не робастны, что частично
объясняется высокой чувствительностью решения к оборачиваемости
матрицы ковариаций.

\subsection*{Что дает теория случайных матриц}
Пусть имеем матрицу доходностей ценных бумаг $ X\in\R^{T\times N} $,
в которой одновременно $ T $ реализаций $ N $ независимых случайных величин,
распределенных совместно нормально с нулевым матожиданием и диагональной
ковариационной матрицей с элементами $ \sigma^2 $. Тогда коваиационная матрица
$ C = T^{-1}X'X $ имеет вектор собственных чисел $ \lambda \in R^{N} $, где
каждое сходится к распределению Марченко-Пастура:
\[
    f(\lambda) = \begin{dcases*}
        \frac{T}{N}\frac{\sqrt{(\lambda_+ - \lambda)(\lambda - \lambda_-)}}{2\pi\lambda\sigma^2}, \text{если } \lambda\in [\lambda_-, \lambda_+]\\
        0, \text{в противном случае}
    \end{dcases*}
\]
причем известно, что матожидание наибольшего собственного числа имеет
формулу
$ \lambda_+ = \sigma^2\left( 1 + \sqrt{\frac{N}{T}} \right)^2 $
а матожидание минимального собственного числа:
$ \lambda_- = \sigma^2\left( 1 - \sqrt{\frac{N}{T}} \right)^2 $.

Определитель случайной матрицы выражается через свои собственные значения следующим
образом:
\[
    \det C = \prod_{k=1}^{K} \lambda_k
\]

Если мы строим портфель из большого количества бумаг, то $ N/T \to 1 $, что
приводит к тому, что $ \lambda_- \to 0 $, а значит ковариационная матрица
доходностей $ N $ бумаг может иметь близкий к нулю определитель.


\section*{Идея}
Использовать подгонку распредления собственных значений к распределению
Марченко-Пастура, чтобы отделить собственные значения, связанные с шумом
от собственных значений, связанных с сигналами.

Пусть $ \{\lambda_n\}_{n=1\dots N} $ - множество всех собственных значений,
отсортированных в порядке убывания, т.е $ \lambda_i > \lambda_{i+1} $. Обозначим
индексом $ i $ пограничный элемент: $ \lambda_i  > \lambda_+$ и
$ \lambda_{i+1} < \lambda_+ $. Тогда мы будем считать все $ \lambda_j $, у которых
$ j > i $ -- собственными значениями шума в данных и преобразуем их
следующим образом $ \lambda_j = \frac{1}{N-j}\sum_{k=i+1}^{N} \lambda_k $ для
всех $ j = i+1\dots N $. Такая замена позволит сохранить след корреляционной
матрицы, равный $ N $.

При заданном разложении $ VW = W \Lambda $, мы формируем
обещшумленную корреляционную матрицу $ C_1 $ как:
\begin{gather*}
    \tilde{C}_1 = W \tilde{\Lambda}W' \\
    C_1 = (\text{diag } \tilde{C}_1)^{-1/2} \tilde{C}_1 (\text{diag } \tilde{C}_1)^{-1/2}
\end{gather*}

\section*{Мотивация для NCO}
Чем больше связаны доходности ценных бумаг, тем менее стабильны
веса, получаемые из формулы выпуклой оптимизации.

Когда $ K $ бумаг образуют кластер, они становятся более подверженными
влиянию общего собственного вектора, из чего следует, что соответствующее
собственное значение объясняет большую часть дисперсии.
Алгоритм:
\begin{enumerate}
    \item \textbf{Кластеризация корреляций}. Находим оптимальное количество
          кластеров (может помочь алгоритм NCO). Для больших матриц с маленьким
          значением $ T/N $, рекоммендуется сначала дешумить корреляционную матрицу
          до этапа кластеризации, используя методы, описанные раннее.
    \item \textbf{Внутри-кластерное взвешивание}.
\end{enumerate}

\section*{Приложение}
Решение методом Лагранжа:
\begin{align*}
    \mathcal{L} = \frac{1}{2}\omega V \omega' + \lambda\left(1 - \omega'a  \right) \to\max_{\omega}
\end{align*}
FOCs:
\begin{align*}
    \frac{\partial \mathcal{L}}{\partial \omega}  & = V\hat{w} - \lambda a = 0 \\
    \frac{\partial \mathcal{L}}{\partial \lambda} & = 1 -  \hat{w}'a = 0       \\
\end{align*}
Ковариационная матрица оборачиваема, значит:
\[
    \hat{w} = \lambda V^{-1}a \to 1 = \lambda a' V^{-1}a \to \lambda = \frac{1}{a'V^{-1}a}
\]
Таким образом:
\[
    \hat{w} = \frac{V^{-1}a}{a'V^{-1}a}
\]
\end{document}
