\documentclass[11pt,a4paper]{article}
\usepackage[margin=1in]{geometry} % full-width
    \topskip        =   20pt
    \parskip        =   10pt
    \parindent      =   0 pt
    \baselineskip   =   15pt
\usepackage{amssymb, amsfonts, amsmath}
\usepackage{bm}
\usepackage{amsmath}
\usepackage{booktabs}  % neatly formatting lines
\usepackage{threeparttable}
\usepackage{graphicx}
\usepackage{caption}
\usepackage{mathtools}
\usepackage{subfig}
\usepackage[shortlabels]{enumitem}

%Russian-specific packages
%--------------------------------------
\usepackage[T2A]{fontenc}
\usepackage[utf8]{inputenc}
\usepackage[russian]{babel}

\newcommand{\bo}[1]{\mathbf{#1}}
\newcommand{\R}{\mathbb{R}}
\newcommand{\E}{\mathbb{E}}
\newcommand{\Y}{X_i'\beta + u_i}
\newcommand{\img}[3]{
    \begin{figure*}[!hbtp]
        \centering
        \caption{#3}
        \includegraphics[scale=#2]{#1}
    \end{figure*}
}
\newcommand{\imgs}[4]{
    \begin{figure}[!hbtp]
        \centering
        \begin{minipage}{.5\textwidth}
            \centering
            \includegraphics[width=.9\linewidth]{#1}
            \captionof{figure}{#3}
        \end{minipage}
        \begin{minipage}{.49\textwidth}
            \centering
            \includegraphics[width=.9\linewidth]{#2}
            \captionof{figure}{#4}
        \end{minipage}
    \end{figure}
}

\begin{document}
\section*{Мотивация}
Хотим решать задачу по поиску оптимальных весов портфеля:
\begin{gather*}
    \frac{1}{2}\omega V \omega' \to \max \\
    \text{s.t}~~ \omega'a = 1
\end{gather*}
Выпуклая оптимизация (см. приложение) дает следующее решение:
\[
    \hat\omega = \frac{V^{-1}a}{a'V^{-1}a}
\]
Как можно видеть из формулы, необходимо оборачивать ковариационную
матрицу доходностей бумаг, из которых мы собираем портфель.
Сама по себе ковариационная матрица случайна, поскольку
мы получаем её оценки на основе данных, используя ММП.
Эмпирика показывает, что оценки весов не робастны, что частично
объясняется высокой чувствительностью решения к оборачиваемости
матрицы ковариаций.

\subsection*{Что дает теория случайных матриц}



Определитель случайной матрицы выражается через свои собственные значения следующим
образом:
\[
    \det X = \prod_{k=1}^{K} \lambda_k
\]

\section*{Приложение}
Решение методом Лагранжа:
\begin{align*}
    \mathcal{L} = \frac{1}{2}\omega V \omega' + \lambda\left(1 - \omega'a  \right) \to\max_{\omega}
\end{align*}
FOCs:
\begin{align*}
    \frac{\partial \mathcal{L}}{\partial \omega}  & = V\hat{w} - \lambda a = 0 \\
    \frac{\partial \mathcal{L}}{\partial \lambda} & = 1 -  \hat{w}'a = 0       \\
\end{align*}
Ковариационная матрица оборачиваема, значит:
\[
    \hat{w} = \lambda V^{-1}a \to 1 = \lambda a' V^{-1}a \to \lambda = \frac{1}{a'V^{-1}a}
\]
Таким образом:
\[
    \hat{w} = \frac{V^{-1}a}{a'V^{-1}a}
\]
\end{document}
